\documentclass[a4paper,11pt]{article}
\usepackage{a4wide}
\usepackage[british]{babel}
\usepackage{hyperref}
\usepackage{url}

\title{Offscale's consensus algorithm}

\author{Vladimir Komendantskiy}

\begin{document}
\maketitle

\abstract{Using prior work on Lachesis consensus \cite{lachesis} and following informal
  implementation notes by Offscale \cite{notes}, I am aiming at defining a gossip-inspired consensus
  algorithm in a way amenable to incremental implementation and formal verification.}


\section{Introduction}


\section{Related work}

Gossip-based consensus algorithms gained wide recognition thanks to cryptocurrency projects such as
IOTA or Hashgraph \cite{hashgraph}. MaidSafe's Parsec \cite{parsec} consensus also uses a gossip
protocol as a base layer on top of which a binary agreement protocol is used to construct a totally
ordered list of blocks. Parsec makes weak synchrony assumptions to guarantee eventual message
delivery.

Out of all the protocols listed above, Hashgraph has received the best coverage with regards to
formal verification \cite{hashgraph-coq} and scrutiny of the implementation
\cite{hashgraph-fud}. This has led to the underlying directed acyclic graph (DAG) data structure
spreading over to other algorithms \cite{parsec, lachesis}. However it should be noted that the
choice of the same DAG data structure is not a requirement for a gossip consensus protocol. More
specifically, even if a gossip protocol uses a DAG to store events and child-parent relations
between those, the number of parents for non-genesis events does not have to be exactly two as in
Hashgraph, and there can be various reasons to prefer a greater number of parents.


\section{Model assumptions}

Any protocol performs up to its theoretic assumptions. Protocols with stronger assumptions -- such
as total message ordering and eventual delivery -- may be easier to define and prove statements
about. When the assumptions don't hold, the statements that rely on those assumptions don't hold
either and the protocol may exhibit arbitrary behaviour. Therefore having strong assumptions may
help in proving statements but those statements will be weak and would cover a small number of
cases.

Offscale's consensus model assumptions are as weak as it is possible for stating protocol
properties. In the model, I assume total asynchrony between participants: messages from a single
sender can be reordered and an arbitrary number of messages can be dropped. Therefore a failed
message sender is indistinguishable from a still operating message sender whose communication links
are down. The only assumption is that a message cannot be received before being sent.

A required protocol property is Byzantine fault tolerance. For a consensus network of size $N$, I
allow the maximum theoretically possible number of Byzantine participants $t$ where $3t < N$.


\section{Basic notions}

\emph{Honest participant} is a consensus network participant adhering to the protocol.

\emph{Malicious participant} is a consensus network participant exhibiting arbitrary, Byzantine behaviour.


\begin{thebibliography}{10}

  \bibitem{hashgraph} Leemon Baird. \emph{The Swirlds Hashgraph Consensus Algorithm: Fair, Fast,
    Byzantine Fault Tolerance.} Swirlds Tech Report SWIRLDS-TR-2016-01. 2016.

  \bibitem{parsec} Pierre Chevalier, Bartolomiej Kamiński, Fraser Hutchison, Qi Ma and Spandan
    Sharma. \emph{Protocol for Asynchronous, Reliable, Secure and Efficient Consensus (PARSEC).}
    2018. \url{https://docs.maidsafe.net/Whitepapers/pdf/PARSEC.pdf}

  \bibitem{lachesis} Sang-Min Choi, Jiho Park, Quan Nguyen and Andre Cronje. \emph{Fantom: A
    scalable framework for asynchronous distributed systems.} 8th February, 2019.
    \url{https://github.com/SamuelMarks/consensus-rough-notes/blob/cd603414cf16526ea8e94ed341d9021eb8e0041f/papers/Fantom__A_scalable_framework_for_asynchronous_distributed_systems.pdf}

  \bibitem{hashgraph-coq} Karl Crary. \emph{Hashgraph consensus aBFT proof in Coq}. 23rd October,
    2018. \url{https://swirlds.com/downloads/hashgraph-coq.zip}

  \bibitem{notes} Maxim Zakharov. \emph{Consensus schema.} 20th September,
    2019. \url{https://github.com/SamuelMarks/consensus-rough-notes/blob/356d9d2fdf8872c23e57f65154522979f522b068/Consensus-schema.md}

  \bibitem{avalanche} Team Rocket. \emph{Snowflake to Avalanche: A Novel Metastable Consensus
    Protocol Family for Cryptocurrencies.} 16th May
    2018. \url{https://ipfs.io/ipfs/QmUy4jh5mGNZvLkjies1RWM4YuvJh5o2FYopNPVYwrRVGV}

  \bibitem{hashgraph-fud} Eric Wall. \emph{Hedera Hashgraph -- Time for some FUD}. 2nd September
    2019.  \url{https://medium.com/@ercwl/hedera-hashgraph-time-for-some-fud-9e6653c11525}

\end{thebibliography}

\end{document}
