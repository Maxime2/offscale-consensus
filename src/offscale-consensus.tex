\documentclass[a4paper,11pt]{article}
\usepackage{a4wide}
\usepackage[british]{babel}
\usepackage{hyperref}
\usepackage{url}

\title{Offscale's consensus algorithm}

\author{Vladimir Komendantskiy}

\begin{document}
\maketitle

\abstract{Using prior work on Lachesis consensus \cite{lachesis} and following informal
  implementation notes by Offscale \cite{notes}, I am aiming at defining a gossip-inspired consensus
  algorithm in a way amenable to incremental implementation and formal verification.}


\section{Introduction}

Replicated state machines underpin blockshain technology. Blockchains whose state contains a unique
chain of blocks follow replicated state machine design to achieve replication of the same chain
across the network.

It should be noted that Offscale's design is one of a \emph{blockchain} since, as Maxim Zakharov put
it in \cite{notes} exactly, ``tasks/problems the consensus layer solves: \dots creating a
\emph{linear order of all events} of a frame based on lamport timestamps of the events and
synchronisation patterns''. To put the quote in context, it should be added that frames are linearly
ordered too. Hence all events that are members of frames are linearly ordered. Therefore an Offscale
blockchain pursues the same fundamental goal as most existing blockchains such as Bitcoin or
Ethereum. However the design of the replicated state machine used by such an Offscale blockchain is
different due to choices made to address inherent problems in today's decentralised mainstream
blockchains: low transaction rate, low transaction confirmation speed and strong synchrony
assumptions.

By increasing the rate at wich transactions can be submitted to the network, and submitted
transactions confirmed, an Offscale blockchain aims to solve a part of the scalability problem which
limits the use of blockchain technology to the ``store of value'' use cases, and to apply thus
improved blockchain technology in domains that strongly rely on high thransaction throughput such as
large-scale heterogenous computational networks commonly referred to as the Internet of Things.

By lifting synchrony assumptions an Offscale blockchain addresses security concerns of synchronous
protocols, Bitcoin included, which require setting an interval between blocks to at least the time
greater than the maximum message delay on the network \cite{rethinking}. The minimum interval
requirement poses a straightforward problem for such protocols: what if the expected maximum message
delay does get exceeded for a majority of participants making them unable to produce or receive
blocks, thereby making it possible for a minority of participants construct the longest (and
therefore valid) chain in the meantime? This is an attack vector on a synchronous blockchain which
can be exploited by a maliciously conspiring adversary. In an Offscale blockchain and in an
asynchronous blockchain in general, this kind of a minority adversary will not be able to mutate the
global state because, even if the majority experiences a network split, a majority vote is still
required in order to mutate the global state, which is why the minority will not be able to make
progress without completely leaving the original network.


\section{Related work}

Gossip-based consensus algorithms gained wide recognition thanks to cryptocurrency projects such as
IOTA or Hashgraph \cite{hashgraph}. MaidSafe's Parsec \cite{parsec} consensus also uses a gossip
protocol as a base layer on top of which a Byzantine binary agreement protocol is used to construct
a chain of blocks. Parsec makes weak synchrony assumptions to guarantee eventual message
delivery. Wavelet \cite{wavelet} seems to follow Parsec on the use of Byzantine binary agreement to
construct a chain of events from a gossip graph, in addition featuring a designated network
management component responsible for keeping participant membership up to date, that is, adding or
removing participants, and managing peer-to-peer connections between those.

Out of all the protocols listed above, Hashgraph has received the best coverage with regards to
formal verification \cite{hashgraph-coq} and scrutiny of the implementation
\cite{hashgraph-fud}. This has led to the underlying directed acyclic graph (DAG) data structure
spreading over to other algorithms \cite{parsec, lachesis}. However it should be noted that the
choice of the same DAG data structure is not a requirement for a gossip consensus protocol. More
specifically, even if a gossip protocol uses a DAG to store events and child-parent relations
between those, the number of parents for non-genesis events does not have to be exactly two as in
Hashgraph, and there can be various reasons to prefer a greater number of parents.


\section{Model assumptions}\label{sec:assumptions}

Any protocol performs up to its theoretic assumptions. Protocols with stronger assumptions --- such
as assumptions of linear message ordering and eventual delivery --- may be easier to define and
prove statements about. When the assumptions don't hold, the statements that rely on those
assumptions don't hold either and instances of the protocol may exhibit arbitrary
behaviour. Therefore having strong assumptions may help in proving statements but those statements
will be weak and would cover only a small number of cases.

Offscale's consensus model assumptions are as weak as it is possible for stating protocol
properties. This allows to capture real-world scenarios involving communication protocols with
unreliable message delivery. In the model, I assume total asynchrony between participants: messages
from a single sender can be reordered and an arbitrary number of messages can be dropped. Therefore
a failed message sender is indistinguishable from a still operating message sender whose
communication links are down. The only assumption is that a message cannot be received before being
sent.

A required protocol property is Byzantine fault tolerance. For a consensus network of size $N$, I
allow the maximum theoretically possible number of Byzantine participants $t$ where $3t < N$.

The permissionless setting is assumed. Any participant is allowed to join or leave anytime, without
requiring an authority for that. The security properties hold for all current participants in the
consensus network. Participants are allowed to have no information about the exact number of
participants in the network.

The protocol satisfies the \emph{consistency property}: at any point in the execution, any two
honest participants have consistent chains. That is, either both their chains are identical, or one
is a prefix of the other.

Due to asynchrony, the protocol does not satisfy the \emph{liveness property}. This property amounts
to any honest protocol participant being able propose to add a transaction to the chain, and if this
transaction then gets added to some honest participant's chain, it is guaranteed to be eventually
incorporated into the chains of all honest participants. Clearly the liveness property is
unattainable in the presence of arbitrary --- possibly infinite --- message delays. Participants may
stall and fail to progress if messages to or from them are delayed or dropped. However this problem
can be dealt with independently of the asynchronous consensus algorithm, for example, by designing a
messaging layer that eliminates arbirtary message delays by detecting and reporting those as
failures.


\section{Basic notions}

\textbf{Honest participant}: a consensus network participant adhering to the protocol.

\noindent
\textbf{Malicious participant}: a consensus network participant exhibiting arbitrary, Byzantine
behaviour.

\noindent
\textbf{Blockchain}, or simply \textbf{chain}: a list of blocks. The chain is the replicated state
of the protocol. All honest protocol participants agree on it in the sense of the consistency
property, see Section \ref{sec:assumptions}.

\noindent
\textbf{Block}: an event signed by a subset of network participants.

\noindent
\textbf{Gossip event}, or simply \textbf{event}: a data structure containing information that a
participant wants to pass to other participants.

\noindent
\textbf{Gossip graph}: a graph data structure with nodes being gossip events and with edges being
child-parent relations. This graph structure satisfies acyclicity: there is no path from any node
leading to itself. In other words, a gossip graph is required to be a DAG and any operations on it
are required to preserve this property.

\noindent
\textbf{Child-parent relation}: a directed acyclic relation between nodes in the gossip graph
connecting events to earlier events that led up to them.

\noindent
\textbf{Genesis event}: a gossip event which is the unique root of the gossip graph. It is the same
across all honest participants.

\noindent
\textbf{Famous event}: a unique event in the gossip graph selected by honest participants according
to a secure protocol which is protected from malicious participants, for example, using a verifiable
delay function or Byzantine binary agreement.


\section{Overview of the protocol}

In the overview below I provide a conceptual presentation of the protocol with minimal details about
how the protocol might be designed at the functional level. Detailed designs may vary widely in the
degree of optimisations or sophistication.

A protocol instance, run by an honest participant, alternates between three tasks indefinitely,
starting from a genesis event:

\begin{enumerate}
\item Add any pending blocks to the chain.
\item Add pending transactions if there are any as a new event in the gossip graph.
\item Synchronise the gossip graph with another participant chosen at random.
\end{enumerate}


\subsection{Construction of the blockchain}

To construct blocks, the gossip graph is divided into sections called rounds, starting from round 0
that contains the unique genesis event $E_0$ with no parents. Every complete round $n$ yields a
unique block $B_n$ that is added to the blockchain. Round $n$ is complete when a new unique famous
event $E_{n+1}$ is computed by honest participants according to the protocol. That new unique famous
event $E_{n+1}$ marks the start of the next round $n + 1$. The new block is then formed from the
transactions contained in the famous event $E_n$ at the start of the current round and from gossip
events that are both children of $E_n$ and ancestors of $E_{n+1}$, not including the event $E_{n+1}$
itself. Those transactions are linearly ordered locally by each honest participant using only the
information in the events of those transactions and rules of the protocol that are the same for all
participants.


\subsection{Adding new transactions to the gossip graph}

New transactions originating at a participant are included in new events in a manner that guarantees
eventual linear ordering of transactions. A trivial example is having one event per transaction
given that the new transactions are linearly ordered between themselves. Every new such event has
the last event created by the participant as one of its parents.


\subsection{Synchronisation of the gossip graph}

Participants make and receive randomised synchronisation requests. Requests themselves are events in
the gossip graph, as well as actions of receiving a request. A request receipt is related to the
request by the child-parent relation. Creating a request receipt is followed by creating a
syncronisation response event by the receiving participant, making the receipt event one of the
parents of the synchronisation response event, and sending a gossip graph update to the sender of
the request.


\begin{thebibliography}{10}

  \bibitem{hashgraph} Leemon Baird. \emph{The Swirlds Hashgraph Consensus Algorithm: Fair, Fast,
    Byzantine Fault Tolerance.} Swirlds Tech Report SWIRLDS-TR-2016-01. 2016.

  \bibitem{parsec} Pierre Chevalier, Bartolomiej Kamiński, Fraser Hutchison, Qi Ma, Spandan
    Sharma. \emph{Protocol for Asynchronous, Reliable, Secure and Efficient Consensus (PARSEC).}
    2018. \url{https://docs.maidsafe.net/Whitepapers/pdf/PARSEC.pdf}

  \bibitem{lachesis} Sang-Min Choi, Jiho Park, Quan Nguyen, Andre Cronje. \emph{Fantom: A scalable
    framework for asynchronous distributed systems.} 8th February, 2019.
    \url{https://github.com/SamuelMarks/consensus-rough-notes/blob/cd603414cf16526ea8e94ed341d9021eb8e0041f/papers/Fantom__A_scalable_framework_for_asynchronous_distributed_systems.pdf}

  \bibitem{hashgraph-coq} Karl Crary. \emph{Hashgraph consensus aBFT proof in Coq}. 23rd October,
    2018. \url{https://swirlds.com/downloads/hashgraph-coq.zip}

  \bibitem{wavelet} Kenta Iwasaki, Heyang Zhou. \emph{Wavelet: A decentralized, asynchronous,
    general-purpose proof-of-stake ledger that scales against powerful, adaptive adversaries}. 26th
    May, 2019.  \url{https://wavelet.perlin.net/whitepaper.pdf}

  \bibitem{rethinking} Rafael Pass, Elaine Shi. \emph{Rethinking Large-Scale Consensus}. 30th
    Computer Security Foundations Symposium (CSF),
    2017. \url{https://ieeexplore.ieee.org/document/8049715}

  \bibitem{avalanche} Team Rocket. \emph{Snowflake to Avalanche: A Novel Metastable Consensus
    Protocol Family for Cryptocurrencies.} 16th May,
    2018. \url{https://ipfs.io/ipfs/QmUy4jh5mGNZvLkjies1RWM4YuvJh5o2FYopNPVYwrRVGV}

  \bibitem{hashgraph-fud} Eric Wall. \emph{Hedera Hashgraph -- Time for some FUD}. 2nd September,
    2019.  \url{https://medium.com/@ercwl/hedera-hashgraph-time-for-some-fud-9e6653c11525}

  \bibitem{notes} Maxim Zakharov. \emph{Consensus schema.} 20th September,
    2019. \url{https://github.com/SamuelMarks/consensus-rough-notes/blob/356d9d2fdf8872c23e57f65154522979f522b068/Consensus-schema.md}

\end{thebibliography}

\end{document}
